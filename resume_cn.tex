% Enable hyperlinks
\setupinteraction
  [state=start,
  style=,
  color=,
  contrastcolor=]
% make chapter, section bookmarks visible when opening document
\placebookmarks[chapter, section, subsection, subsubsection, subsubsubsection, subsubsubsubsection][chapter, section]
\setupinteractionscreen[option=bookmark]
\setuptagging[state=start]

% use microtypography
\definefontfeature[default][default][script=latn, protrusion=quality, expansion=quality, itlc=yes, textitalics=yes, onum=yes, pnum=yes]
\definefontfeature[smallcaps][script=latn, protrusion=quality, expansion=quality, smcp=yes, onum=yes, pnum=yes]
\setupalign[hz,hanging]
\setupitaliccorrection[global, always]
\setupbodyfontenvironment[default][em=italic] % use italic as em, not slanted
\usemodule[simplefonts]
\setmainfontfallback[DejaVu Serif][range={greekandcoptic, greekextended}, force=yes, rscale=auto]
\setupwhitespace[medium]

\setuphead[chapter]            [style=\tfd,header=empty]
\setuphead[section]            [style=\tfc]
\setuphead[subsection]         [style=\tfb]
\setuphead[subsubsection]      [style=\bf]
\setuphead[subsubsubsection]   [style=\sc]
\setuphead[subsubsubsubsection][style=\it]

\setuphead[chapter, section, subsection, subsubsection, subsubsubsection, subsubsubsubsection][number=no]

\definedescription
  [description]
  [headstyle=bold, style=normal, location=hanging, width=broad, margin=1cm, alternative=hanging]

\setupitemize[autointro]    % prevent orphan list intro
\setupitemize[indentnext=no]

\setupfloat[figure][default={here,nonumber}]
\setupfloat[table][default={here,nonumber}]

\setupthinrules[width=15em] % width of horizontal rules


\starttext

\section[吴-晓]{吴 晓}

Maybe you need my
\useURL[url1][http://tinyslik.github.io/resume][][English
resume]\from[url1]

\subsection[万金油c图形工程师]{万金油C++图形工程师}

\startitemize[packed]
\item
  电话: \useURL[url2][tel://86-15861657693][][86-15861657693]\from[url2]
\item
  邮件:
  \useURL[url3][mailto:tinysilk@hotmail.com][][tinysilk@hotmail.com]\from[url3]
\item
  博客(墙外):
  \useURL[url4][http://tinyslik.github.io][][tinyslik.github.io]\from[url4]
\item
  博客(墙内):
  \useURL[url5][http://tinyslik.coding.me][][tinyslik.coding.me]\from[url5]
\item
  Github:
  \useURL[url6][http://github.com/TinySlik][][TinySlik]\from[url6]
\stopitemize

\subsection[简介]{简介}

一个有四年开发经验的小司机。
开发过很多款上线手游项目,上架了appStroe,googlePlay以及国内大量的杂鱼渠道;

\subsection[特长]{特长}

能胜任android(java),ios/mac(OC),win32,linux
单平台或者跨平台C++的开发工作,对css,html5,js均有良好的掌握度,能够进行前端开发;
对openGL web/ES/2.x+管线有良好的掌握度。对3d图形数学有良好的认识和基础。
长期使用cocos2d-x游戏引擎,基于win32,QT有丰富的C++架构经验,对工具编写也有良好的习惯。

\subsection[工作经验]{工作经验}

\subsubsection[cocos2d-x-开发工程师-在-mochang有限公司]{{\bf cocos2d-x
开发工程师} 在
\useURL[url7][https://www.mochang.net/][][MoChang有限公司]\from[url7]}

{\em 2014.1 - 2015.6}

1.在公司主要从事cocos2d-x游戏的制作和维护工作,使用C++重构代码,配置更多的功能模块.
2.作为从事生物科技工作的我来说,这是一次挑战,我学习到了很多C++基本的语法及表驱动,插件架构等常用的语言设计模式。补习了很多程序及系统的基础原理,对linux系统的学习了一段时间,能独立编写小型的c++服务器。
3.在公司我工作了一年半,上架了一些比较简单的欧美儿童游戏,有大量的功能模块重用和维护。
4.期间也独立完成很多 ios 及安卓端的 sdk包括广告的和功能的接入。
5.制作小型游戏制定文档并按照需求文档在较短的时间内提交上线。
6.由于团队小,所以大量的与美术的协调工作都由我独立完成,团队的策划担任高层领导角色,团队具有比较高的自由度。

\subsubsection[cc-开发工程师-在-南京光辉互动网络科技有限公司熊大叔儿童教育]{{\bf C/C++
开发工程师} 在
\useURL[url8][https://bie-plc.com/][][南京光辉互动网络科技有限公司]\from[url8]/\useURL[url9][https://www.biemore.com/zh-cn/index.html][][熊大叔儿童教育]\from[url9]}

{\em 2015.6 - 2016.2}

1.一线开发游戏项目,对多个产品线不同语言的项目代码进行检查和修改重构,使用代码包括包括js,Lua,C++。
2.实现shader技术优化和难点(水纹,模糊)。
3.负责一些产品最后阶段的代码调试和检查工作,接入一些商业代码,sdk。
4.具体在这里同时修改了多个项目的代码,包括捕鱼达人3D(Lua),宝宝熊教育系列(JavaScript),并完成了最后的上线。

\subsubsection[前端工程师兼职小团队项目管理-在-南京触控科技办事处]{{\bf 前端工程师兼职小团队项目管理}
在
\useURL[url10][http://www.chukong-inc.com/][][南京触控科技办事处]\from[url10]}

{\em 2016.3 - 2016.10}

1.参与3.x的cocos2d-x引擎代码的修改和维护,上架一些以前的成品代码游戏。
2.培训新员工,培训git,lua,quick-cocos部分内容。 3.修改和上线u3d项目.
4.管理部分员工并上线一些小的游戏项目类似手机三国志,棋牌等等。

\subsubsection[opengl工程师兼图形算法工程师-在-无锡威莱斯科技有限公司]{{\bf openGL工程师兼图形算法工程师}
在
\useURL[url11][http://vless.net/][][无锡威莱斯科技有限公司]\from[url11]}

{\em 2016.11 - 2017.8}

1.在开源框架openNI的基础上定制化公司需求的视频流架构用于算法分析。
2.使用openCV完成轮廓提取,去背景等简单图像算法问题。
3.完成openGL的图形界面软件,包括传感器信息的展示,量程,摄像机,彩虹色,录像,播放,多路同步分析等等软件框架下的数据基础工作。
4.上线公司的产品,测试新的软件产物并完成文档。

\subsubsection[主要语言技能]{主要语言技能}

注:\crlf
+:使用过,了解语法,能够上手修改现有的浅层代码逻辑\crlf
++:全盘学习过,理解语言全盘的语法,对语言的相关设计理念有认识,对相关特性够熟练运用
+++:相关语言的基本设计模式能够完全应用于实践,能够胜任框架的搭建

\startitemize[packed]
\item
  \useURL[url12][http://www.cplusplus.com/][][C++]\from[url12]+++
\item
  C++
\item
  \useURL[url13][http://www.lua.org/][][Lua]\from[url13]+++
\item
  \useURL[url14][https://www.java.com/zh_CN/][][java]\from[url14]+
\item
  \useURL[url15][https://developer.apple.com/][][objective-C]\from[url15]+
\item
  \useURL[url16][https://www.microsoft.com/net/][][C\#]\from[url16]++
\item
  \useURL[url17][https://www.glslsandbox.com/][][shader]\from[url17]++
\item
  \useURL[url18][https://www.javascript.com/][][javaScript]\from[url18]+++
\item
  \useURL[url19][http://developers.whatwg.org][][HTML]\from[url19]+++
\item
  \useURL[url20][http://www.w3.org/Style/CSS/Overview.en.html][][CSS]\from[url20]++
\item
  \useURL[url21][https://www.python.org/][][python]\from[url21]++
\item
  \useURL[url22][http://www.ruby-lang.org/zh_cn/][][Ruby
  (Rake)]\from[url22]++
\item
  \useURL[url23][http://www.uml.org/][][UML]\from[url23]++
\stopitemize

\subsubsection[格式类型使用]{格式类型使用}

注:\crlf
+:了解并修改过小于5次。
++:全盘学习过,熟练使用语法书写内容或修改配置。
+++:查看过底层格式实现原理,并修改使用过自己的衍生格式。

\startitemize[packed]
\item
  \useURL[url24][http://daringfireball.net/projects/markdown][][Markdown]\from[url24]++
\item
  \useURL[url25][https://www.xml.com/][][XML]\from[url25]++
\item
  \useURL[url26][http://www.json.org.cn/][][Jason]\from[url26]++
\item
  \useURL[url27][https://github.com/Winnerhust/inifile2][][ini]\from[url27]++
\item
  \useURL[url28][http://wixtoolset.org/][][wix toolset]\from[url28]+
\stopitemize

\subsubsection[框架]{框架}

注*:\crlf
+:使用过,了接口用法,能够快速使用。
++:模块化地学习过,理解框架下原理,能够熟练进行相关优化。
+++:查看学习过框架源码,修改框架内部实现并商用化过相关的代码。

\startitemize[packed]
\item
  \useURL[url29][https://www.opengl.org/][][OpenGL
  (web,ES,glut,glew,glsl\ldots{})]\from[url29]++
\item
  \useURL[url30][http://opencv.org/][][OpenCV]\from[url30]+
\item
  \useURL[url31][https://www.qt.io/][][QT]\from[url31]++
\item
  \useURL[url32][http://www.boost.org/][][Boost
  (shared_ptr,λ,tuple,thread\ldots{})]\from[url32]++
\item
  \useURL[url33][http://www.cocos2d-x.org/][][Cocos2d-x]\from[url33]+++
\item
  \useURL[url34][https://hexo.io/][][hexo]\from[url34]++
\item
  \useURL[url35][https://www.openai.com/][][OpenAI]\from[url35]+
\stopitemize

\subsubsection[软件]{软件}

注*:\crlf
+:使用过,能够完成常规的工具功能
++:熟练使用,包括快捷键和高度自定义的功能
+++:查看过软件源码,修改衍生过相关的商用版本

\startitemize
\item
  IDE:\useURL[url36][http://developer.apple.com][][Apple
  Xcode]\from[url36]++/\useURL[url37][https://www.visualstudio.com/][][VisualStudio]\from[url37]++/\useURL[url38][http://www.android-studio.org/][][AndroidStudio]\from[url38]+
\item
  MAKE: make / Cmake / Rake / XMake +
\item
  \useURL[url39][http://git-scm.com][][Git]\from[url39]++
\item
  \useURL[url40][http://svn.apache.org][][Subversion]\from[url40]+
\item
  \useURL[url41][https://www.sourcetreeapp.com/][][SourceTree(win)]\from[url41]/\useURL[url42][https://www.git-tower.com/][][Tower(mac)]\from[url42]++
\item
  \useURL[url43][http://www.gnu.org/software/grub/][][grub/grub2]\from[url43]++
\item
  \useURL[url44][http://apple.com/macosx][][Mac OS
  X]\from[url44]++/\useURL[url45][http://ubuntu.com][][Ubuntu
  Linux]\from[url45]++
\item
  \useURL[url46][http://www.sublimetext.com][][Sublime
  Text]\from[url46]++
\item
  \useURL[url47][http://www.vim.org][][Vim]\from[url47]++
\item
  \useURL[url48][http://www.gnu.org/software/bash/][][bash]\from[url48]++
  / \useURL[url49][http://www.zsh.org][][zsh]\from[url49]++
\item
  \useURL[url50][http://jetbrains.com/webstorm][][WebStorm]\from[url50]+
\item
  \useURL[url51][http://johnmacfarlane.net/pandoc][][Pandoc]\from[url51]+
\item
  \useURL[url52][https://github.com/doxygen/doxygen][][Doxygen]\from[url52]+
\item
  \useURL[url53][http://www.latex-project.org/][][Latex]\from[url53]+
\item
  \useURL[url54][http://wiki.nginx.org][][Nginx]\from[url54]+
\item
  \useURL[url55][http://mysql.com][][MySQL]\from[url55]+
\stopitemize

\subsection[教育]{教育}

\useURL[url56][][][扬州大学 动物医学]\from[url56], 2008 - 2012

\subsection[兴趣]{兴趣}

\startitemize[packed]
\item
  玩卡牌游戏(炉石),至今未上传说\ldots{}
\item
  听音乐,玩一些器材,中西内外新旧都听
\item
  养猫,叫毛毛,一只英短蓝色的小母猫
\item
  弹吉他,准备周末去卖艺
\item
  逛github,知乎,Bilibili 找基佬玩耍
\item
  另外已婚未育
\stopitemize

\subsection[下载]{下载}

以下是我的多种格式下的简历,如有需要请自行下载:

\useURL[url57][https://github.com/TinySlik/resume/raw/master/resume_cn.docx][][doc]\from[url57]

\useURL[url58][https://github.com/TinySlik/resume/raw/master/resume_cn.epub][][ePub]\from[url58]

谢谢对我的关注.

©2016 \useURL[url59][http://tinyslik.coding.me/resume][][Tiny
Oh]\from[url59]. All rights reserved.

\stoptext
